\documentclass[notheorems]{beamer}

\usepackage{kip-math, kip-quantum, mathtools}
\usepackage[backend=biber]{biblatex}
\usepackage{geometry, enumerate, hyperref}
\usepackage{graphicx, caption, subcaption}
\usepackage{bookmark}

\addbibresource{../sources.bib}

\title{Quantum Error Correction and Stabilizer Codes}
\author{River McCubbin}
\date{\today}

\begin{document}
\frame{\titlepage}

\begin{frame}
\frametitle{Overview}
\begin{enumerate}
\item Quantum Computing \pause
\item Quantum Errors \pause
\item Quantum Error Correction
\end{enumerate}
\end{frame}

\begin{frame}
\frametitle{Quantum Computing}
\begin{center}
\begin{tabular}{c|c|c}
& Data & Operations\\
\hline
Classical & Bits & Gates\\
\hline
Quantum & Vectors & (Unitary) Operators
\end{tabular}
\end{center}
\end{frame}

\begin{frame}
\frametitle{Quantum States}
Qubits:
$$\ket{\psi} = \alpha \ket{0} + \beta \ket{1}$$
where
\begin{align*}
\ket{0} &= \begin{bmatrix} 1 \\ 0 \end{bmatrix} & \ket{1} &= \begin{bmatrix} 0 \\ 1 \end{bmatrix}
\end{align*}
unit vectors in $\bb{C}^2$, i.e. $\alpha^2 + \beta^2 = 1$.
\end{frame}

\begin{frame}
\frametitle{Quantum Operations}
Quantum Gates (Unitary Operators):
        $$ U(\ket{\psi}) = \alpha U \ket{0} + \beta U \ket{1}$$
such that
        $$U^*U = UU^* = I$$
in $M_2(\bb{C})$.
\end{frame}

\begin{frame}
\frametitle{(Complex) Tensors}
$\ket{\psi}, \ket{\phi}, \ket{\gamma} \in \bb{C}^2$:\\ \pause
bilinear: 
        \begin{align*}
        (a \ket{\phi}) \tensor \ket{\psi} &= a ( \ket{\phi} \tensor \ket{\psi}) = \ket{\phi} \tensor (a \ket{\psi})\\
        (\ket{\phi} + \ket{\psi}) \tensor \ket{\gamma} &= \ket{\phi} \tensor \ket{\gamma} + \ket{\psi} \tensor \ket{\gamma}
        \end{align*} \pause
symmetric: $\ket{\phi} \tensor \ket{\psi} = \ket{\psi} \tensor \ket{\phi}$
\end{frame}

\begin{frame}
\frametitle{Tensor Example}
$\ket{\psi} = \alpha \ket{0} + \beta \ket{1}$,\\
$\ket{\phi} = a \ket{0} + b \ket{1}$,\pause
\begin{align*}
\ket{\psi} \tensor \ket{\phi} 
&= (\alpha \ket{0} + \beta \ket{1}) \tensor (a \ket{0} + b \ket{1})\\ 
&= \alpha \ket{0} \tensor a \ket{0} + \alpha \ket{0} \tensor b \ket{1} + \beta \ket{1} \tensor a \ket{0} + \beta \ket{1} \tensor b \ket{1}\\ 
&= a \alpha \ket{00} + b \alpha \ket{01} + a \beta \ket{10} + b \beta \ket{11} 
\end{align*}
\end{frame}

\begin{frame}
\frametitle{Multi-Qubit Gates}
\begin{itemize}
\item Tensors of single-qubit gates $X \tensor X, I \tensor X$ \pause
\item Other: $\CNOT, \SWAP$
\end{itemize}
\end{frame}

\begin{frame}
\frametitle{Common Gates}
\begin{align*}
X \ket{0} &= \ket{1} & X \ket{1} &= \ket{0}\\
Z \ket{0} &= \ket{0} & Z \ket{1} &= -\ket{1}\\
H \ket{0} &= \ket{+} & H \ket{1} &= \ket{-}\\ 
\CNOT \ket{00} &= \ket{00} & \CNOT \ket{01} &= \ket{01}\\
\CNOT \ket{10} &= \ket{11} & \CNOT \ket{11} &= \ket{10}
\end{align*}
where 
\begin{align*}
\ket{+} &= \frac{1}{\sqrt{2}} ( \ket{0} + \ket{1}) & \ket{-} &= \frac{1}{\sqrt{2}} (\ket{0} - \ket{1})
\end{align*}
\end{frame}


\begin{frame}
\frametitle{Quantum Errors}
\begin{itemize}
\item $X$ errors: $X\ket{0} = \ket{1}, X\ket{1} = \ket{0}$ \pause
\item $Z$ errors: $Z\ket{0} = \ket{0}, Z\ket{1} = -\ket{1}$
\end{itemize}
\end{frame}

\begin{frame}
\frametitle{No-Cloning}
There does not exist an operator $U$ such that
        $$U \ket{\psi} = \ket{\psi} \tensor \ket{\psi}$$
\end{frame}

\begin{frame}
\frametitle{Bit Flip Code}
\begin{description}
\item[Encoding:]
        With input state $\ket{\psi} = \alpha \ket{0} + \beta \ket{1}$, encode via
        $\CNOT (\ket{\psi} \tensor \ket{0}) \tensor \ket{0} = (\alpha \ket{00} + \beta \ket{11}) \tensor \ket{0}$\pause
\item[Channel:]
        Applies an $X$ gate to one of the first two qubits with probability $p$. \pause

\item[Syndrome:]
        First, map the third qubit to $\ket{+}$ by applying an $H$ gate.\\ \pause
        Detect the syndrome by applying a controlled $ZZ$ gate on the first two qubits controlled by the third.\\ \pause
        Map the third qubit back to the standard basis by applying another $H$ gate.\\ \pause
        The syndrome is 0 if no errors are detected, 1 if errors are detected. \pause

\item[Correction:]
        This code is not capable of correcting errors. \pause

\end{description}
Notice that this code has (in classical terms) distance 1!
\end{frame}

\begin{frame}
\frametitle{Bit Flip Circuit}
\begin{quantikz}
\ket{\psi} & \ctrl{1} \slice{Encoding} & \gate[2]{Channel} \slice{}     &       & \gate[2]{Z \tensor Z}      & \slice{Syndrome} &\\
\ket{0} & \gate{X}      &                       &       &       &       &\\
\ket{0} &               &                       & \gate{H} & \ctrl{-1} & \gate{H} & \meter{}
\end{quantikz}
\end{frame}

\begin{frame}
\frametitle{Example}
        $X$-Error detection example.\\ \pause
        $X$-Error correction example.
\end{frame}

\begin{frame}
\frametitle{Conclusion}
A similar method can be used to correct $Z$-errors (attached file).\\ \pause
We can extend both of these to correct arbitrary numbers of errors, at the cost of much larger codes.
\end{frame}

\begin{frame}
\frametitle{Thank You!}

\nocite{*}
\printbibliography

\end{frame}

\end{document}
